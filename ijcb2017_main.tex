\documentclass[10pt,twocolumn,letterpaper]{article}

\usepackage{btas}
\usepackage{times}
\usepackage{epsfig}
\usepackage{graphicx}
\usepackage{amsmath}
\usepackage{amssymb}


\usepackage{gensymb}
\usepackage{url}
\usepackage{subcaption}
\usepackage{hhline}

\usepackage[breaklinks=true,bookmarks=false]{hyperref}

% Include other packages here, before hyperref.

% If you comment hyperref and then uncomment it, you should delete
% egpaper.aux before re-running latex.  (Or just hit 'q' on the first latex
% run, let it finish, and you should be clear).
%\usepackage[pagebackref=true,breaklinks=true,letterpaper=true,colorlinks,bookmarks=false]{hyperref}

%\btasfinalcopy % *** Uncomment this line for the final submission

\def\btasPaperID{****} % *** Enter the IJCB Paper ID here
\def\httilde{\mbox{\tt\raisebox{-.5ex}{\symbol{126}}}}

% Pages are numbered in submission mode, and unnumbered in camera-ready
\ifbtasfinal\pagestyle{empty}\fi
\begin{document}

%%%%%%%%% TITLE
\title{Deep Convolutional neural network for Fingerprint Pattern Classification}

\author{First Author\\
Institution1\\
Institution1 address\\
{\tt\small firstauthor@i1.org}
% For a paper whose authors are all at the same institution,
% omit the following lines up until the closing ``}''.
% Additional authors and addresses can be added with ``\and'',
% just like the second author.
% To save space, use either the email address or home page, not both
\and
Second Author\\
Institution2\\
First line of institution2 address\\
{\tt\small secondauthor@i2.org}
}

\maketitle
\thispagestyle{empty}

%%%%%%%%% ABSTRACT
\begin{abstract}
   Fingerprints are broadly categorized into five pattern class types.
   Fingerprint pattern classification is useful for quick exclusions and 
   therefore reducing the search space by allowing for search and 
   comparison only within same pattern class type.
   Fingerprint pattern classification can be done by visual examination of
   shape and characteristics of regions of a fingerprint; or using classification 
   algorithms such as neural networks or K-nearest neighbors that are trained 
   to recognize some specified patterns.  To our knowledge, all of the current 
   fingerprint pattern classification algorithms require some processing of 
   fingerprint images to extract features, for example, estimate orientation flow 
   and detect singularities.   In this paper, we present deep learning approach 
   for automated fingerprint pattern type classification without explicit characterization  
   of fingerprint or any feature extraction. 
   {
	\color{red}
	This method uses deep convolutional neural network(deep ConvNet) as feature extractor and a (Support Vector Machine) SVM as classifier.
	Our results show that the proposed approach outperforms the state-of-the-art approaches, achieving $98.61\%$ accuracy on NIST SD14 and $100\%$ accuracy on NIST SD4.
}
\end{abstract}

\section{Introduction}
%!TEX root = main.tex


Fingerprints are ridge and valley patterns presented on the surface of human fingertips.
%
Fingerprints are used to recognize humans for applications such as verifying an identity claim (\textit{i.e.}, one-to-one search to  unlock a smartphone, for example), or identification (\textit{i.e.}, one-to-many search to find a suspect of a crime, for instance).
%
Typically, to query a fingerprint, a system needs to search and compare the query print with the fingerprints stored in its reference (or enrolled) database.  The size of a reference database can be from thousands to hundreds of millions of subjects, depending on the application. For example, the Aaddhar project in India has enrolled 111,98,29,743 persons as of February 18, 2017 \cite{aaddhaar}.  
%
As the size of the database grows, the number of comparisons to be made for identification purposes grow, so does the computation time.
%
To mitigate this problem, most fingerprint recognition algorithms first classify a fingerprint into a basic pattern type and then perform fingerprint matching within fingerprints of that type.
%
The major five fingerprint pattern types used today are an extension of the three pattern types (whorl, loop, and arch) introduced by Henry Faulds (Henry classification system \cite{henry1905classification}) and Sir Francis Galton \cite{galton1892} in late 19th century. These five pattern types are: arch, left loop, right loop, tented arch and whorl, see Figure\ref{fig.fingerprint_classes}.  
%
The Fingerprint Source Book \cite{nijSourceBook} defines each of these patterns as follows: 
%
Arch is a pattern type in which the friction ridges enter on one side of the impression and flow, or tend to flow, out the other side with a rise or wave in the center. 
%
Tented arch is a pattern type that possesses either an angle, an upthrust, or two of the three basic characteristics of the loop. 
%
Loop is a pattern type in which one or more friction ridges enter upon one side, recurve, touch or pass an imaginary line between delta and core and flow out, or tend to flow out, on the same side the friction ridges entered. Loops can be left slant loops (or left loops), in which the pattern flows to the left in the impression; or right slant loops (right loops), in which the pattern flows to the right in the impression.
%
Whorl is a fingerprint pattern type that consists of one or more friction ridges that make, or tends to make, a complete circuit, with two deltas, between which, when an imaginary line is drawn, at least one recurving friction ridge within the inner pattern area is cut or touched. 

As mentioned above, to manage the computation load, large scale fingerprint identification algorithms employ multi-stage matching whose first step is often filtering based on fingerprint pattern type.\footnote{Some of more recent fingerprint classification algorithms use continuous classifications instead of discrete classes. These continuous spectrums are beyond the scope of this paper because of their non-intuitive and proprietary nature as they are developed for a dedicated recognition algorithm.} As such the accuracy of the fingerprint classification algorithm largely influences the identification accuracy. An error in finger pattern classification will propagate throughout the system, and ultimately result in an recognition error. 
Challenges in fingerprint pattern classification include: 
%
1) quality of fingerprints, characterizing poor quality images are particularly difficult,
%
2) small inter-class dissimilarity and small intra-class similarity, for example, tented arch and loop may look similar; 
%
3) ambiguities in some pattern class labels, some fingerprints can be classified into multiple classes, or different classes by different fingerprint experts.

In this paper we propose an automated fingerprint pattern classification that unlike previous finger pattern classifiers, does not require feature extraction.  Specifically, we trained a deep convolutional neural network to take a fingerprint image as an input and classify it into one of the five pattern class types of a) Arch; b) Tented Arch; c) Left Loop; d) Right Loop; or e) Whorl. 

The rest of the paper is organized as follows.  Section \ref{sec_motivtn} overviews previous work. 
Section \ref{sec_method} details our technical approach. 
We describe our experimental set up, data and results in Section \ref{sec_exp}.
Finally, we conclude in Section \ref{sec_con} and present our suggested way forward.

\begin{figure}[!ht]
	\begin{center}
		\includegraphics[width=8cm]{fig/Fingerprint_classes.png}
	\end{center}
	\caption{Examples of fingerprint classes\cite{cao2013fingerprint}: (a) Arch, (b) Tented Arch,  (c) Left Loop, and  (d) Right Loop.  Because arch and tented arch only accounts for a small portion in human, most often they are combined into one class.} 
	\label{fig.fingerprint_classes}
\end{figure}




%-------------------------------------------------------------------------

%\section{Related Work}
%\input{sec_relatedwork.tex}

\section{Previous work}
\label{sec_motivtn}
%
Fingerprint pattern classification is a well studied field and there are many works reported in literatures.  An overview of some related work follows.
%
Karu and Jain \cite{karuJain1996} presented a rule-based classifier based on extracting singular points. 
%
Fitz and Green \cite{FitzGreen1996} used a Hexagonal Fourier Transform to classify fingerprints into whorls, loops and arches. 
%
Jain \textit{et al.} \cite{JainSalil1999} used a bank of Gabor filters to compute a feature vector (FingerCode) and then used a K-nearest neighbor classifier and a set of neural networks to classify a feature vector into one of the five fingerprint pattern classes.
%
Cappelli \textit{et al.} \cite{cappelli1999} partitioned a fingerprint directional image into ``homogeneous'' connected regions according to the fingerprint topology, resulting in a synthetic representation which is then used as a basis for the classification.
%
Bernard \textit{et al.} \cite{Bernard2001} used the Kohonen topologic map for fingerprint pattern classification. 
%
Kai Cao \textit{et al.}\cite{cao2013fingerprint} proposed to extract fingerprint orientation feature and used a hierarchical classifier for classification.
%
Ruxin Wang \textit{et al.} \cite{wang2014fingerprint} also used orientation filed as features. %By adopting a stacked autoencoder, they achieved 93.1\% in four-class classification.
%%


Previous works mostly consist of singularity points (core and delta) detection or extracting features such as ridge and orientation flow, or using human markups (or handcrafted features) as the basis for pattern type classification. 
%
Therefore, the accuracy of these methods depends on the goodness (or utility) of the selected features and the precision of the feature extraction portion of the pattern classification algorithms. Both are sensitive to the noise and the variations of the gray-scale level of the input image.  
%
Using handcrafted features can improve performance.  However, in addition to being burdensome and time consuming, accuracy of handcrafted features cannot be guaranteed due to the existence of noise and poor image quality. 
%
Their repeatability and reproducibility cannot be guaranteed either, due to inter- and intra-examiners variations \cite{fbiBlackbox}.  
%
Our approach differs from the previous works in the sense that we use fingerprint images instead of features as input. Convolutional neural network (CNN) has the capability of learning features and it can be directly applied on raw images. CNN exhibits powerful classification capability in many areas\cite{lecun2015deep}\cite{szegedy2016rethinking}.
%
Different from common CNN-based approach, we employ a Support Vector Machine (SVM) on top of deep ConvNet as classifier for prediction.





%\section{Proposed research}
%\input{sec_plan.tex}
%\label{sec_plan}

%\section{Dataset}
%\input{sec_dataset.tex}
%\label{sec_dataset}

\section{Methodology}
\label{sec_method}
%!TEX root = main.tex

Figure\ref{fig.method} illustrates our technical approach.   
%
The top row shows the steps of the training process.  First training images are preprocessed using data augmentation techniques to increase data diversity. 
The augmented data are fed into deep ConvNet for training. The trained deep ConvNet then serves as a feature extractor for a SVM which uses the deep features from the trained deep ConvNet as input, and is trained to classify the pattern type.
%
During the testing, shown in the bottom row of Figure\ref{fig.method}, testing images are fed into the trained deep ConvNet for deep features extraction. These intermediate deep features are used as input features to the trained SVM. The output of the SVM is final pattern class prediction.


\begin{figure}[!ht]
	\begin{center}
		\includegraphics[scale=0.38,clip=true,trim = 50mm 55mm 50mm 50mm]{fig/figs/method_overview.pdf}
	\end{center}
	\caption{Overview of proposed approach.} 
	\label{fig.method}
\end{figure}

%-------------------------
\subsection{Deep ConvNet Architecture}
\label{sec_cnn}
Our proposed network architecture (Figure \ref{fig.cnn_arch})  is based on deep residual network proposed by Kaiming He \textit{et al.}\cite{he2016deep}. Deep residual  network has been proven to outperform other deep plain networks because it addresses the degradation problem by reformulating the layers as learning residual functions instead of learning unreferenced functions. 
%
Table\ref{tab.cnn_params} summarizes the details of our network. The input size of our deep ConvNet is $512\times512$ and the number of channels is $1$.
The first two layers of our network are convolutional layers which has 96 $7\times7$ filters and the stride is 2. The third layer has 64 $7\times7$ filters and the stride is 2. 
%
\textit{conv4}, \textit{conv5}, \textit{conv6} and \textit{conv7} are composed of residual building blocks. Specifically, there is a max pooling layer before  \textit{conv4}. The parameters inside the brackets specify the residual building block size. 
%
The multiplier after bracket specifies the multiplicity of the that block in that layer.  More details and explanations regarding network architecture can be found in \cite{he2016deep}. 
%
The global pooling layer in \textit{conv8} generates $1\times1,2048$ output and the last layer uses 5 $1\times1$ filters to generate the final prediction. 
%
The number of parameters of proposed deep ConvNet is $24.26$ million. 
%
We use Relu\cite{nair2010rectified} as intermediate activation function.
%

The novelty of our network is that we use $512\times512$ as network input size. This preserves the detail of fingerprint to the extent possible. 
%
However, the larger the size of input images, the larger the computational cost.  We empirically searched for the optimal input image size and we found that the performance drops as the size gets smaller, and it drops significantly for images smaller than $224\times224$ -- as shown in Figure\ref{fig.resize_examples}, down-sampling the images to $224\times224$, results in loss of information content of prints and reduces the clarity of ridge details and deltas and cores. 
%
Therefore we decided against significant down sampling.  
%
We also ruled out cropping or segmenting the fingerprint portion of an image, that is removing background and non-fingerprint portion of images to get smaller images, because of its requiring to employ a segmentation algorithm and reliance on the accuracy of the segmentation algorithm -- recall that one of our objective has been to avoid fingerprint processing and characterization.
%
After careful visual inspection of fingerprint images of different sizes, we decided to use $512\times512$ as input size for the following reasons. 
%
First, sufficient fingerprint detail is preserved at this size. Second, it is a reasonable size for rolled fingerprints. Finally, it is the original image size of one of the datasets we used, which is NIST SD4.
%
To remedy the huge computational cost and high memory usage during the training, we added \textit{conv1} and \textit{conv2} with stride 2 to down-sample the input images. As shown in Table.\ref{tab.cnn_params}, after \textit{conv2}, the feature map size is $128\times128, 96$. So, the spatial size is reduced (from $512 \times 512$ to $128 \times 128$) and spatial information is stored in the increased channels (from $1$ to $96$).

%
\begin{figure}[!ht]
	\begin{center}
		\includegraphics[scale=0.28,clip=true,trim = 20mm 15mm 10mm 10mm]{fig/figs/resize_examples.pdf}
	\end{center}
	\caption{Example images from NIST SD14 with different spatial sizes. The top row images are $512\times512$ and the bottom row are $224\times224$. Ridges of top images are more distinguishable than that of bottom images. } 
	\label{fig.resize_examples}
\end{figure}
%
%
\begin{figure*}[!ht]
	\begin{center}
		\includegraphics[scale=0.65,clip=true,trim = 20mm 65mm 40mm 65mm]{fig/figs/cnn_arch.pdf}
	\end{center}
	\caption{Architecture of proposed CNN.} 
	\label{fig.cnn_arch}
\end{figure*}


\begin{table}[!ht]
	\centering
	\caption{Detail of proposed deep ConvNet. The format is inspired by \cite{he2016deep}}
	\label{tab.cnn_params}
	\begin{tabular}{l}
		\includegraphics[scale=0.45,clip=true,trim = 78mm 5mm 70mm 5mm]{fig/figs/cnn_table.pdf}
	\end{tabular}
\end{table}

%-------------------------
\subsection{Data Augmentation}
\label{sec-dataAug}
Fingerprint images exhibit a wide range of location, rotation, brightness and contrast. To enhance the generalization ability of our ConvNet, we adopt data augmentation techniques to increase the data diversity.
	
To augment training dataset, we applied below augmentation techniques:

\begin{enumerate}

	\item Random Cropping. The input images are first resized to $532 \times 532$. We randomly cropped a $512\times512$ region from the resized images.
	\item Random Rotation. We randomly rotate the input images by $\omega$ degrees where $\omega \sim uniform (-30\degree, 30\degree)$.
	\item Random Brightness.  Random brightness change is performed on the input images. The gray scale of the input images $I$ are change to $I + \delta$ where $\delta$ is  sampled from $uniform (-50, 50)$.
	\item Random Contrast. We randomly change the contrast of images. The contrast factor is sampled from $uniform (0.4, 1.6)$.

\end{enumerate}

%-------------------------
\subsection{SVM}
\label{sec_svm}
The deep ConvNet only serves as a feature extractor and we use a non-linear SVM as classifier. The kernel is radial basis function(RBF). The gamma of RBF kernel  is set to be $\frac{1}{n}$ where $n$ is the feature dimensionality. The penalty for error term $C$ is set to be $1.0$. 
%
We use the output of  conv7\_x as features. Therefore, each sample is represented by a feature vector $x \in \mathbb{R}^d$ where $d=4*4*2048=32768$. The output of SVM is the predicted label $\hat{y} $ indicating one of the fingerprint pattern class types.
%










\section{Experiments}
\label{sec_exp}
%!TEX root = main.tex

\subsection{Dataset}

In this project, we  use NIST Special Database 4 \cite{nist-db-4} and NIST Special Database 14 \cite{nist-db-14} for our experiments. 
%
The NIST SD4 contains $2000$ 8-bit gray scale fingerprint image pairs, totally 4000 images.
%
The size of each image is $512\times512$ and each image is classified using one of the five following classes: Arch, Left and Right Loops, Tented Arch, Whorl.
%
Each of the five classes has 400 pairs(800 images). Each of the fingerprint pairs are two completely different rollings of the same fingerprint.

%
The NIST SD14 contains $27000$ 8-bit gray scale fingerprint image pairs. There are 2700 subjects in this dataset and each subject has 10 fingerprint samples pairs. The size of each image is  $768\times832$. To fit in our network, we centrally crop $768\times768$ from the samples and resize them into $512\times512$. 
%
The distribution of fingerprint classes is as shown in Table.\ref{tab.sd14_dist}. We can see that unlike NIST SD4, Arch and Tented Arch samples are only a small portion of the NIST SD14.


\begin{table}[!ht]
	\centering
	\caption{Class Distribution of NIST SD14.}
\label{tab.sd14_dist}
	\begin{tabular}{|c|c|c|c|c|}
		\hline
		\textbf{Arch} & \textbf{Left Loop} & \textbf{Right Loop} & \textbf{\begin{tabular}[c]{@{}c@{}}Tented \\ Arch\end{tabular}} & \textbf{Whorl} \\ \hline
		3.6\% & 31.9\% & 30.5\% & 3.2\% & 30.8\% \\ \hline
	\end{tabular}
\end{table}

\subsection{Experimental Setup}
We use a i7-5930K desktop with 32GB memory and a Nvidia GTX TITAN X GPU for experiments.
%
Typically, we use Tensorflow 1.0.1 as the deep learning library and Adaptive Moment Estimation(Adam\cite{kingma2014adam}) as the optimization algorithm. The learning rate is 0.0001. We also use $\ell_2$ regularization with 0.0001 weight decay rate. The batch size is 32. 
%
We evaluate our approach on NIST SD4 and NIST SD14 respectively. Each experiment is trained for $20k$ steps.


%
For NIST SD14 experiments, we use the samples of $80\%$ subjects for training, totally 2160 subjects with $43200$ images. Among these $432000$ images, $36$ of them have labels that do not belong to the 5 classes. These $36$ images are discarded. 
The remained data of $20\%$ subjects are used for testing, totally $10800$ images. $9$ of these images are discarded due to the same reason above.
%

For NIST SD4 experiments,  we adopt two evaluation protocols. 
%
The first protocol is cross-sample for fair comparison with other works, where we use all the first samples in each fingerprint pair as training set and all the second samples as testing set.
%
The second protocol is cross-finger, where we use 50\% fingers for training and 50\% for testing to ensure the same finger does not existing in training and testing set at the same time. To improve the performance for NIST SD4, We use NIST SD14 data to pre-train the model.
 

In addition to 5-class fingerprint classification, we also evaluate our SVM performance on 4-class fingerprint classification because 4 class classification are also used in other studies. 
%
To achieve 4-class classification, we merge Tented Arch class into Arch class when training SVM.

\begin{figure*}[!ht]
	\begin{subfigure}[b]{0.25\textwidth}
		\centering
		\includegraphics[width=\linewidth]{fig/figs/confusion_matrix_svm_sd14.pdf}
		\caption{SVM for NIST SD14 }
		\label{fig.cnf_matrix_5class.svm_sd14}
	\end{subfigure}%
	\begin{subfigure}[b]{0.25\textwidth}
		\centering
		\includegraphics[width=\linewidth]{fig/figs/confusion_matrix_net_sd14.pdf}
		\caption{ConvNet for NIST SD14 }
		\label{fig.cnf_matrix_5class.net_sd14}
	\end{subfigure}%
	\begin{subfigure}[b]{0.25\textwidth}
		\centering
		\includegraphics[width=\linewidth]{fig/figs/confusion_matrix_svm_sd4_cross_subject.pdf}
		\caption{SVM for NIST SD4 }
		\label{fig.cnf_matrix_5class.svm_sd4}
	\end{subfigure}%
	\begin{subfigure}[b]{0.25\textwidth}
		\centering
		\includegraphics[width=\linewidth]{fig/figs/confusion_matrix_net_sd4_cross_subject.pdf}
		\caption{ConvNet for NIST SD4 }
		\label{fig.cnf_matrix_5class.net_sd4}
	\end{subfigure}
	\caption{Confusion Matrices for 5-class classification}\label{fig.cnf_matrix_5class}
\end{figure*}

\begin{figure}[!ht]
	\begin{subfigure}[b]{0.25\textwidth}
		\centering
		\includegraphics[width=\linewidth]{fig/figs/confusion_matrix_svm_sd14_4class.pdf}
		\caption{SVM for NIST SD14 }
		\label{fig.cnf_matrix_4class.svm_sd14}
	\end{subfigure}%
	\begin{subfigure}[b]{0.25\textwidth}
		\centering
		\includegraphics[width=\linewidth]{fig/figs/confusion_matrix_svm_sd4_4class_cross_subject.pdf}
		\caption{SVM for NIST SD4}
		\label{fig.cnf_matrix_4class.svm_sd4}
	\end{subfigure}%

	\caption{Confusion Matrices for 4-class classification}\label{fig.cnf_matrix_4class}
\end{figure}

\subsection{NIST SD14 result}
The result for NIST SD14 is shown in Table\ref{tab.SD14_result}.
%
In addition to report SVM performance, we also report the performance when ConvNet is used as classifier. 
%
As we can see, both ConvNet and SVM achieve the same accuracy ($0.9861$) for 5-class classification. 
%
However, ConvNet performs slightly better in terms of average precision, recall rate and F1 score.
%
For 4-class classification, the 4-class SVM achieves $0.9875$ accuracy.

For 5-class classification, the confusion matrix is shown in Figure.\ref{fig.cnf_matrix_5class.svm_sd14} and Figure.\ref{fig.cnf_matrix_5class.net_sd14}.
For 4-class classification, the confusion matrix is shown in Figure.\ref{fig.cnf_matrix_4class.svm_sd14}. 
%
As we can see, the number of Arch and Tented Arch samples are relatively small compared to other classes.
%
However, our proposed approach can still achieve high accuracy despite the unbalanced distribution of fingerprint types.
%
Though computation based on Figure.\ref{fig.cnf_matrix_5class.svm_sd14}, we can see that Tented Arch achieves the lowest precision (0.959) and recall rate(0.950) due to lack of training samples and label ambiguity. 
%
Based on Figure.\ref{fig.cnf_matrix_4class.svm_sd14}, both the precision and recall rate of Arch increases to $0.98$, indicating many mis-classified samples can be eliminated by combing Arch and Tented-Arch classes.


\begin{table}[!ht]
	
	\centering
	\caption{ Experiment results for NIST SD14. In column 4, 5 and 6, we also report the average precision, recall and F1 score for all predicted classes. }
	\label{tab.SD14_result}
	\scalebox{0.87}{
	\begin{tabular}{|c|c|c|c|c|c|}
		\hline
		\textbf{method} & \textbf{\begin{tabular}[c]{@{}c@{}}\# of \\ classes\end{tabular}} & \textbf{accuracy} & \textbf{\begin{tabular}[c]{@{}c@{}}average \\ precision\end{tabular}} & \textbf{\begin{tabular}[c]{@{}c@{}}average\\  recall\end{tabular}} & \textbf{\begin{tabular}[c]{@{}c@{}}average \\ F1 score\end{tabular}} \\ \hline
		ConvNet & 5 & 0.9861 & 0.9843 & 0.9793 & 0.9817 \\ \hline
		SVM & 5 & 0.9861 & 0.9822 & 0.9781 & 0.9801 \\ \hline
		SVM & 4 & 0.9875 & 0.9869 & 0.9867 & 0.9868 \\ \hline
	\end{tabular}
}
\end{table}

\subsection{NIST SD4 Result}

The result for NIST SD4 is shown in Table\ref{tab.SD4_result}.
%
We have three observations from Table\ref{tab.SD4_result}.
%
First, in both protocol, our proposed SVM performs better than ConvNet not only in accuracy but also in average precision, recall rate and F1 score. 
%
In cross-sample protocol, the accuracy of 5-class SVM is $0.9275$, which is $0.006$ higher than 5-class ConvNet. In cross-finger protocol, the accuracy of 5-class SVM is $0.912$, $0.014$ higher than 5-class ConvNet.
%
Second, there is a performance drop in cross-finger compared to cross-sample. For 5-class ConvNet, the accuracy drops $0.023$. For 5-class SVM, the accuracy drops $0.015$. For 4-class SVM, the accuracy drops $0.011$. 
%
SVM suffers the smaller performance drop than ConvNet if cross-finger protocol is used.
%
The performance drop is small, indicating the generalization ability of our proposed method.
%
Third, the accuracy is improved if Tented Arch and Arch are combined into one-class. The accuracy of 4-class SVM is $0.022$ higher than 5-class SVM in cross-sample and $0.027$ higher in cross-finger. This indicates many mis-classified samples can be eliminated by combing Arch and Tented-Arch classes, as in NIST SD 14.

For 5-class classification using cross-finger protocol, the confusion matrices are shown in Figure.\ref{fig.cnf_matrix_5class.svm_sd4} and Figure.\ref{fig.cnf_matrix_5class.net_sd4}.
%
For 4-class classification using cross-finger protocol, the confusion matrices are shown in Figure.\ref{fig.cnf_matrix_4class.svm_sd4}.
%
We can see that both in Figure.\ref{fig.cnf_matrix_5class.svm_sd4} and Figure.\ref{fig.cnf_matrix_5class.net_sd4}, the Tented Arch class achieves the lowest precision ($91.8\%$) and lowest precision rate ($94.0\%$) among five classes. 
%
In later experiments, we can see that the mis-labeled samples are ambiguous can be eliminated by introducing the second labels.

In NIST SD4, around $17\%$ of the samples are ambiguous and are labeled with two classes. Many existing works report their best performance based on these additional labels.
%
We also evaluate our approach using the additional $17\%$ two labels to compare with other methods and the results are reported in Table.\ref{tab.SD4_result_two_labels}. 
%
The training procedure remains the same where we only use one label for training. 
%
When testing, for those $17\%$ samples, as long as the prediction for the test sample matches one of the two labels, the test sample is considered 
As we can see, a significant performance gain is obtained after the additional $17\%$ are used.
%
For cross-sample,our proposed ConvNet achieves 0.9535 accuracy, which is 0.032 higher than before.
%
The proposed 5-class and 4-class SVMs achieve 1.0 accuracy and is the best among all the methods.
%
For cross-finger, our proposed ConvNet achieves 0.945 accuracy, which is 0.046 higher than before.
%
The proposed 5-class and 4-class SVMs still achieve 1.0 accuracy in this protocol.


\begin{table}[!ht]
	\centering
	\caption{ Experiment results for NIST SD4. In column 4, 5 and 6, we also report the average precision, recall and F1 score for all predicted classes. }
	\label{tab.SD4_result}
		\scalebox{0.87}{
	\begin{tabular}{|c|c|c|c|c|c|}
		\hline
		 \textbf{method} & \textbf{\begin{tabular}[c]{@{}c@{}}\# of \\ classes\end{tabular}} & \textbf{accuracy} & \textbf{\begin{tabular}[c]{@{}c@{}}average \\ precision\end{tabular}} & \textbf{\begin{tabular}[c]{@{}c@{}}average\\  recall\end{tabular}} & \textbf{\begin{tabular}[c]{@{}c@{}}average \\ F1 score\end{tabular}} \\ \hline
		 \multicolumn{6}{|c|}{\textbf{Cross-Sample}}      \\ \hline
		ConvNet & 5 & 0.9215 & 0.9225 & 0.9215 & 0.9217 \\ \hline
		SVM & 5 & 0.9275 & 0.9325 & 0.9275 & 0.9288 \\ \hline
		SVM & 4 & 0.9495 & 0.9576 & 0.9459 & 0.9514 \\ 
\hhline{|======|}
\multicolumn{6}{|c|}{\textbf{Cross-Finger}}      \\ \hline
		ConvNet & 5 & 0.8985 & 0.8991 & 0.8986 & 0.8987 \\ \hline
SVM & 5 & 0.9120 & 0.9132 & 0.9117 & 0.9123 \\ \hline
SVM & 4 & 0.9390 & 0.9452 & 0.9357 & 0.9403 \\ \hline
		
\end{tabular}}
\end{table}


\begin{table}[!ht]
	\centering
	\caption{Experiment results for NIST SD4 with two labels.}
	\label{tab.SD4_result_two_labels}
	\begin{tabular}{|c|c|c|c|}
		\hline
		\textbf{method} & \textbf{\# of classes} & \textbf{accuracy} & \textbf{protocol} \\ \hline
		ConvNet & 5 & 0.9535 & cross-sample \\ \hline
		SVM & 5 & 1.0 & cross-sample \\ \hline
		SVM & 4 & 1.0 & cross-sample \\ \hline
		\cite{cao2013fingerprint} & 5 & 0.959 & cross-sample \\ \hline
		\cite{cao2013fingerprint}& 4 & 0.972 & cross-sample \\ \hline
		\cite{wang2014fingerprint} & 4 & 0.980 & not-spepcified \\ \hline
		ConvNet & 5 & 0.945 & cross-finger \\ \hline
		SVM & 5 & 1.0 & cross-finger \\ \hline
		SVM & 4 & 1.0 & cross-finger \\ \hline
	\end{tabular}
\end{table}


\subsection{Discussion}
%
From experiment results we can see that our proposed approach can successfully perform fingerprint type classification on raw fingerprint images without the need of extracting hand-crafted features.
%
Our proposed approach achieves the highest classification accuracy on both NIST SD14 and NIST 4 dataset to the best of our knowledge. 
%
Experiment results show that using Deep ConvNet as a feature extractor and train a SVM on top of the ConvNet can bring further performance gain compared to a standalone Deep ConvNet.
%
Tented Arch and Arch fingerprints contributes the most error rate among the five classes.
%
Many mis-classified labels can be corrected by combining Tented Arch and Arch classes into one class or using additional labels.
%
We also collect some mis-classification cases on NIST SD14  and show them in Figure\ref{fig.fails}.
%
As we can see, in 

\begin{figure*}[!ht]
	\begin{center}
		\includegraphics[scale=0.53,clip=true,trim = 5mm 60mm 5mm 45mm]{fig/figs/fail.pdf}
	\end{center}
	\caption{Mis-classification examples on NIST SD14. The title of each example is \textit{Prediction}(\textit{Ground Truth})} 
	\label{fig.fails}
\end{figure*}





\section{Conclusion and Future Work}
\label{sec_con}
%!TEX root = main.tex

We presented a deep learning approach for automated fingerprint pattern type classification. We designed a deep ConvNet based on residual network. To preserve as much fingerprint details as possible, the input image  is designed to be  $512\times512$ pixels and we used two early convolutional layers to reduce the computational cost.  We did data augmentation to improve data diversity.  The deep ConvNet serves as a feature extractor and a SVM is trained as the final classifier. Experiment results show that our proposed method achieves high accuracy comparable to the state-of-the-art approaches using raw images and without any feature extraction.  We did not report reliability (or uncertainty) associated with accuracy of our method.  We will do that as part of our future work by using different sampling of test and train sets.  We would  like to have tested our method on a larger datasets, but we were limited by the availability of ground-truth pattern class labels.  We are seeking solutions for that.  Also, as part of future work, we like to examine the robustness of our method particularly when dealing with images of poor quality.
Other future works include using more advanced deep networks and ensemble techniques to fuse multiple classifiers.



{\small
\bibliographystyle{ieee}
\bibliography{ijcb2017_main}
}

\end{document}
